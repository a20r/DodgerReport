
\begin{document}

\chapter{Objectives}

\label{chapter:objectives}

\section{Primary}

The main objective of this work is to develop an algorithm that generates a
quantitatively safe trajectory for a robot through an uncertain dynamic
environment by utilizing information about how dynamic obstacles are going to
move in the future. This can be simply stated as determining the curvature that
minimizes the line integral over the dynamic cost distribution for a given set
of dynamic obstacles.

\begin{equation}
    J(C, A) = \int \limits_{C} \exp{\Big(P(x, y, t_0, t_m, A) + 1\Big)} \,
    \mathrm{d}s
    \label{eq:objective}
\end{equation}

Eq.~\ref{eq:objective} describes the objective function, $J$, that needs to be
minimized with respect to the curvature in order to determine the safest path
through the environment. In Eq.~\ref{eq:objective}, the function $P$ is the
cost surface for a given time interval and set of obstacles. More description
about $P$ is given in Sec.~\ref{sec:cost} and is formally defined in
Eq.~\ref{eq:prob}. More precisely, the objective of this work to develop an
algorithm that will provide an approximate solution to
Eq.~\ref{eq:argminobjective} which will return the minimum cost path through a
environment for a given set of obstacles. The solution is described in
Sec.~\ref{sec:design_planner}

\begin{equation}
    \Gamma(A) = \argmin{C} \, J(C, A)
    \label{eq:argminobjective}
\end{equation}


\section{Secondary}

The main secondary objective for this work is to show that the proposed
solution can provide safer paths than standard planners such as potential
fields by leveraging information about how the obstacles move through the
environment. Quantitative and qualitative experiments have been conducted that
provide evidence that the proposed solution does indeed produce safer paths
based on the safety metrics that have been devised for this work. These results
are shown in Ch.~\ref{chapter:results}.

\end{document}
