
\begin{document}

\chapter{Software Development Framework}

\label{chapter:software}

% \section{Development Methodology}
% 
% Since this project is primarily research, an agile development methodology was
% used. There were no set requirements for the project, and instead the author
% was encouraged to produce results towards a goal. The solution was devised by
% determining a set of intermittent goals and possible solutions. As you can see
% in Ch.~\ref{chapter:design}, four different solutions were designed and three
% were implemented, each using a different approach to solve the problem at hand.
% The development feedback loop worked by first listing the objectives and goals
% for the project, i.e. generating safe paths through dynamic environments. Once
% an algorithm was designed that the author was confident could solve this
% problem, it was implemented and tested both quantitatively and qualitatively.
% If the behaviour of the robot did not resemble the desired behaviour, the
% solution was either augmented to possibly change the behaviour of the robot, or
% a new solution was devised.
% 
% \section{Source Control}
% 
% As a method of storing and updating the codebase, source control was used. In
% particular, GitHub was used to store the code, its revisions, releases, and
% notes. Along with the main codebase, GitHub was used to store the scripts
% created for running tests and experiments and kept track of the experimental
% results during development. Only working and compilable code was pushed to the
% Git repository and since there was only one developer, a continuous integration
% server was not used. Instead, before a commit could be pushed to the server,
% the code had to be compilable and pass a set of sanity checks. These sanity
% checks consisted of checking whether roadmaps could still be generated and if
% the basic operations on the data structures still functioned properly.

\end{document}
