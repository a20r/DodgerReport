
\begin{document}

\chapter{Design}

\label{chapter:design}

As with any research project, many different attempts were made to come up with
a solution to the objectives state in Ch.~\ref{chapter:objectives}. Three
different techniques were developed in sequence to try and provide a solution
that best matched the sought behaviour for the planner. The first attempt was a
simple potential field that would take into account the predicted trajectories
of the obstacles, and leverage this information to provide safer paths. The
second attempt included generating a probabilistic roadmap in relative
space-time and using stock graph search algorithms such as Dijkstra's algorithm
and Edmonds' algorithm to derive low costs paths through the environment. The
last attempt, and the most successful is a planner that uses a two dimensional
probabilistic roadmap to sample the search space and then uses Best First
Search to expand nodes in space-time to determine the minimum cost path through
the dynamic environment. These three attempts are described individually and
more detail in this section.

\section{Potential Fields}

Using potential fields was an initial attempt to plan through uncertain dynamic
environments since they are frequently used to plan around dynamic obstacles
due to their reactive behaviour~\cite{pf, wallar_taros_2013,
wallar_ssci_2014_boids}.  The difference between the standard potential field
implementations and the one developed for this project was that the repulsive
obstacle field at a position $(x, y)$ was proportional to the cost distribution
at $(x, y)$ for a given time interval.  This is shown more formally in
Eq.~\ref{eq:badpotential}.

\begin{equation}
    U_{\Var{rep}}(p, t_0, t_m, A) = k \cdot P(p_x, p_y, t_0, t_m, A)
    \label{eq:badpotential}
\end{equation}

In Eq.~\ref{eq:badpotential}, the function $P$ is defined in Eq.~\ref{eq:prob},
and $k > 0$ is a constant.  The attractive potential field was kept the same as
the standard potential field implementation for robotic motion planning as
shown in Eq.~\ref{eq:badpotentialattr}

\begin{equation}
    U_{\Var{att}}(p, g) = \frac{c}{||p - g||^2 + \epsilon}
    \label{eq:badpotentialattr}
\end{equation}

In Eq.~\ref{eq:badpotentialattr}, $\epsilon$ is a constant such that $0 <
\epsilon < c$ and is used to ensure that the function does not have a
singularity and $c$ is a scaling constant such that $c > 0$. The potential
field planner would use the sum of these two fields to measure the potential
through the environment in order to eventually reach the goal by successively
moving to the area within the robot's sensing radius that had the minimal
potential. Algo.~\ref{algo:pf} describes more formally how the potential field
planner generates a path through the environment.

\begin{algorithm}[ht]

    \caption{$\Function{PF}(q, g, O, A, R)$}

    \label{algo:pf}
    \begin{algorithmic}[1]
        \setcounter{ALC@line}{0}
        \vspace*{1mm}

        \STATE $q_{\Var{min}} \leftarrow q$
        \STATE $p_{\Var{min}} \leftarrow \infty$
        \STATE $\theta \leftarrow 0$
        \WHILE {$\theta \leq 2\pi$}
            \STATE $q' \leftarrow q + \delta t \cdot s \cdot
            \Function{Rot}(\theta)$
            \STATE $p \leftarrow U_{\Var{rep}}(q', O \cup A)
            + U_{\Var{att}}(q', g)$
            \IF {$p < p_{\Var{min}}$}
               \STATE $p_{\Var{min}} \leftarrow p$
                \STATE $q_{\Var{min}} \leftarrow q'$
            \ENDIF
            \STATE $\theta \leftarrow \theta + \delta \theta$
            \FORALL {$a \in A$}
                \STATE $\Function{Step}(a)$
            \ENDFOR
        \ENDWHILE

        \IF {$||q_{\Var{min}} - g|| < R$}
            \RETURN $\{p_{min}\}$
        \ENDIF

        \RETURN $\{q_{\Var{min}}\} \cup \Function{PF}(q_{\Var{min}}, g, O, A, R)$
    \end{algorithmic}
\end{algorithm}

Through some qualitative testing, these types of potential fields were still
leading the robot into unsafe areas and caused the robot to collide with the
dynamic obstacles regardless of velocity of the obstacles and the velocity of
the robot. After some manipulation of the constants used for the repulsive and
attractive potentials, there was only a nominal improvement which lead the
author to move towards sampling based motion planning techniques which are
outlined in Sec.~\ref{sec:stroadmap} and Sec.~\ref{sec:dodgermethod}.

\section{Space-time Roadmap}

\label{sec:stroadmap}

The second attempt at devising a solution to the primary objective in
Ch.~\ref{chapter:objectives} was to create a three dimensional probabilistic
roadmap that can capture the connectivity of a two dimensional surface in
space-time. A probabilistic roadmap (PRM) is a method of creating an
undirected, weighted graph, $(V, E, W)$, that represents the connectivity of
the search space by randomly sampling points and connecting them such that if
$(i, j) \in E$, then both $i$ and $j$ must not collide with an obstacle, $||i -
j|| \leq d$ where $d$ indicates the maximum distance away connected nodes can
be from one another, and there must not be a collision with any obstacle along
the edge from $i$ and $j$~\cite{prm}. For the attempted space-time PRM, each
node would be a vector, $(x, y, t)$, which represents a two dimensional
location, $(x, y)$, at a certain absolute time $t$, the graph was directed such
that for an edge, $(i, j)$, $i_t < j_t$, and instead of randomly sampling a
point in the environment, a node in the graph would be randomly selected and
propagated forward in time by some random change in time such that the
constraints for the roadmap are still satisfied.

% put roadmap construction algorithm here for st-map

With the generated roadmap, the first thought was to use a graph search
algorithm such as A*~\cite{astar} or Dijkstra's algorithm~\cite{dijkstra} to
find the path through the environment that had the lowest overall weight.  The
weight for an edge, $(i, j)$, defined as the line integral over the cost
surface for a given set of dynamic obstacles with a time interval of $[i_t,
j_t]$.  The notion of a cost distribution is described in
Ch.~\ref{chapter:methodology} and the formal equation for this line integral is
given in Eq.~\ref{eq:cost}.  This space-time roadmap approach yielded mixed
results.  The robot would sometimes evade the obstacles, but with the
incidence, the planner would lead the robot directly into a collision with a
dynamic obstacle. After some testing, it was discovered that since graph search
algorithms seek to find the path with minimum combined weight through the
graph, these algorithms are biased to return paths with a smaller number of
vertices.  This is because paths with a larger number of vertices will have a
higher overall weight. Since shortest path algorithms try to minimize this
overall weight, paths which may be safer but may take longer not be returned by
these algorithms.

To overcome this, instead of searching over the entire graph, the search could
be occur over the minimum spanning tree of the graph.  Since the roadmap is a
directed graph, Edmonds' algorithm~\cite{edmonds} was used since other minimum
spanning tree algorithms such as Kruskal's algorithm~\cite{kruskal} and Prim's
greedy algorithm~\cite{prim} only work on undirected graphs. Using the minimum
spanning tree would minimize the maximum cost associated with a path from the
initial configuration to the goal configuration thus moving the robot away from
high cost areas in space-time. Through qualitative analysis, this approach was
shown to still lead the robot to high cost areas and even into collisions with
obstacles.

This approach using a three dimensional probabilistic roadmap did not work in
practice regardless of the search algorithm because of the number of nodes that
need to be sampled in space-time in order for it to be effective.  The roadmap
indicates where in the environment the robot is able to travel to from a
starting location in space-time. If the number of nodes is too small the, the
goal may not even be in the graph, and thus the robot will never reach it.
Also, the less nodes in the graph the less optimal the generated path is, but
the more nodes added to the graph, the more computationally difficult it
becomes to search. Lastly, the main flaw with this approach is that biases the
sampling to areas that already have a high sample density and therefore may not
sample nodes in the goal area without having a high number of nodes in the
graph.

\section{Probabilistic Roadmap With Best First Search}

After consideration for other sampling based motion planning techniques such as
rapidly exploring random trees~\cite{rrt} and expansive space trees~\cite{est}
the author chose to explore using a custom graph search algorithm over a two
dimensional probabilistic roadmap due to the lack of ability for classical
sampling based techniques to deal with time-dependent edge costs. The idea was
to use a probabilistic roadmap to capture the connectivity of the two
dimensional environment and to generate a search tree through the roadmap that
encodes the temporal information for each point in a tree node and is therefore
able to account for time-dependent edge costs. This graph search algorithm is a
temporal analogue to best-first search which tries to expand the best current
node in a search tree based on some heuristic~\cite{bestfs}.

\label{sec:dodgermethod}.

\end{document}
