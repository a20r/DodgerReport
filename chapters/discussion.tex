
\begin{document}

\chapter{Discussion}

\label{chapter:discussion}

\section{Evaluation}

\section{Future Work}

There are many possible improvements to this project that could not be
accomplished in the given time frame. This section addresses these future
improvements and how they can impact the performance and robustness of the
project.

\subsection{Pruning the Search Tree}

Currently, the only mechanism used to prune the search tree is the priority of
the nodes being expanded. This allows some nodes in the probabilistic roadmap
not to be expanded thus limiting the overall size of the tree. However, more
complex machine learning techniques could be used such as limiting the density
of the search tree in certain areas and not expanding nodes in a geometric area
that have consistently shown to have higher costs. Pruning the search tree
could have drastic result in reducing the computational time needed to find a
safe path because less nodes would need to be expanded and searched. Another
approach at limiting the size of the search tree could be to use another
heuristic that could provide some information about how close a certain node is
to reaching the goal. This would find paths to the goal more quickly because
the search is more directed. There would have to be some constraints used in
this heuristic, for instance, its influence in the priority of a node in the
queue because the safeness of a path would be sacrificed in return for
decreased computational time. Likewise, decreasing the average degree of a node
in the probabilistic roadmap whilst keeping it one connected component would
limit the size of the search tree and thus computational time. This could be
done by using alternative roadmap construction methods such as connecting the
$k$-nearest neighbours instead of all nodes within a given radius.

\subsection{Variable Speed and Wait Time}

There are two constants in this work, the speed and wait time, which if made
made dynamic would increase the robustness of the algorithm. It is assumed that
the robot will travel at a constant speed at all times unless it is waiting for
a set period of time. However, there may be some reasons for increasing or
decreasing the speed of the robot during runtime. For instance, if the robot
were to also try and save energy, it may not to maintain a high speed, but only
increase its speed when it is needed to avoid coming into contact with
obstacles in order to improve safety and minimize energy usage. Also,
determining dynamically how long the robot should wait at a given location
instead of having a set length of time would increase the robustness of the
approach. By using a constant wait time, it is assumed that that interval is
the best possible amount of time to stay stationary. In reality, in some
situations, the robot will need to wait for longer than others. This is
currently not incorporated into the approach and would be a valuable addition.

\subsection{Increasing Robot Dimensionality}

In the current work, it is assumed that the robot is a two dimensional point in
space. This means that the robot occupies at any given time a certain point in
space and that point in space exclusively. Many robot systems operate in higher
dimensional spaces such as snake robots and unmanned aerial vehicles. Snake
robots operate in the same dimension as the number of links it has. Unmanned
aerial vehicles operate in three dimensional space because they are able to
fly. By incorporating higher dimensional vehicle models into the algorithm,
paths for more classes of robots can be generated and thus makes the approach
more robust.

\subsection{Real World Experiments}

Since this is a project is concerned with moving a physical robot through an
environment, an important piece of future work would be to conduct experiments
using a physical robot in a real-world scenario. This would mean adapting the
path generated for the robot in to a list of control inputs that would account
for the physics of the robot to move it to the goal. Since the project provides
a sequence of spatio-temporal waypoints, a specialized controller can be
created for different mobile robots that could follow the waypoints generated.
It may also be beneficial to utilize the control model for the given robot into
the planning. So instead of the result of the planning algorithm returning a
sequence of waypoints, it could directly return a sequence of control inputs
that could be fed into a robot's controller to move it to the goal. Another
important component of implementing the system on a mobile robot would be the
active prediction of obstacle movements. In this work, it is assumed that there
is an external system being used that can determine where obstacles are going
to move in the future. This system would either have to be implemented or an
existing one used in order for the algorithm to work in real-time. This motion
prediction system could either be its own component in an open loop system
(like those found in warehouses), or the motion prediction could occur directly
on the robot. If it is the latter, novel algorithms are going to need to be
developed that are able to predict the motion of obstacles relatively quickly
due to the limits set by the robot's processing power.

\subsection{More Complex Models of Obstacle Behaviour}

Obstacle behaviour in the current work is very simple. Obstacles are assumed to
move along their prescribed trajectory with a set amount of uncertainty and are
not affected by the actions of the robot or the other obstacles near it. It is
also assumed that the obstacles will only follow one trajectory function and
the current formulation does not account for obstacles with dynamic trajectory
equations. The goal of future work would be to formalize a more complete model
of the way an obstacle moves through the environment by incorporating reactive
behaviour and multiple trajectory equations with associated spatio-temporal
probabilities of switching between them. The reactive behaviour will encompass
how the obstacle will react when other objects, either other obstacles or the
robot being planned, move toward the obstacle or interfere with its trajectory.
Likewise, the obstacles can be assumed to also be autonomous agents with their
own planning algorithm. Accounting for this behaviour will result in a more
comprehensive and realistic model of dynamic obstacles.

\subsection{Multi-robot Coordination}

Another extension to the current algorithms provided in this work would be to
allow them to plan for multiple robots simultaneously. The current
implementation only provides a path through a stochastic dynamic environment
for a single robot. Many applications of robotics need multiple robots to move
through the environment safely and at the same time. These algorithms are
usually called swarming algorithms. By extending the current algorithms to find
the $k$-safest paths through the environment in space-time, paths can be
generated for multiple robots and thus safely plan the movements of a swarm to
its goal.

\section{Conclusions}

\end{document}
