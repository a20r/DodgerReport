
\begin{document}

\chapter{Results}

\label{chapter:results}

This chapter presents the results from the experimentation of the algorithm.
The metrics used and the experimental setup is described in
Ch.~\ref{chapter:experimentalsetup}. This chapter will describe the empirical
data gathered for the safety and computational time of the algorithm developed
in Sec.~\ref{sec:design_planner}. This chapter will also compare the proposed
approach with a more standard approach, potential fields, using the empirical
data gathered. For all of the plots presented, the data for the algorithm
developed in this work is shown on the left and the data for the standard
potential fields planner is shown on the right. Also note that this chapter
only surveys the results for Scene 1 because the data was very similar for all
of the scenes. The plots for the other scenes is shown in the appendix.

\section{Safety}

Providing an empirical evaluation of the safety of the developed planner is
incredibly important because if it does not provide quantifiable safe paths
through the environment, the main objective was not satisfied. This section
presents the data gathered for the three safety metrics with the experimental
setup described in Ch.~\ref{chapter:experimentalsetup}. The three safety
metrics are the minimum distance at any given time during the execution of the
path to a dynamic obstacle, the maximum cost along the path given by the cost
distribution discussed in Eq.~\ref{eq:cost}, and the average cost along the
path.

Fig.~\ref{fig:plot_min_distance} shows how the mean minimum distance changes as
a function of the set speed of the robot and the amount of noise introduced to
the trajectories of obstacles. The larger the minimum distance, the safer the
path is. From the figure, it is evident that as the speed increases, so does
the mean minimum distance Dodger. The amount of noise injected into the
obstacle trajectories, $\epsilon$, does not seem to have any major effect on
the minimum distance for any speed. For potential fields, as the speed
increases, the increase in the minimum distance is not as drastic and the noise
does not affect the overall safety given this metric. The differences between
these two plots stem from the fact that a potential fields planner is purely
reactive and Dodger generates a path by looking ahead. This means that the
potential fields planner could move the robot through the path of an obstacle
thus decreasing the minimum distance and increasing the possibility of a
collision. Dodger will avoid moving the robot through the trajectory of an
obstacle and thus is more likely to have a larger minimum distance along the
path. For the gathered empirical data shown in
Fig.~\ref{fig:plot_min_distance}, Dodger provided paths with a higher overall
minimum distance than potential fields and as the speed of the robot increased,
the minimum distance for the paths generated by Dodger out performed those
generated by potential fields for this metric.

\begin{figure}[h!]
    \centering
    \includegraphics[width=0.48\linewidth]{figs/planner_mean_min_distance_0}
    \includegraphics[width=0.48\linewidth]{figs/pf_mean_min_distance_0}
    \caption{Plots showing how the average minimum distance to the obstacles
    changes as the speed increases for various amounts of obstacle position
    uncertainties}
    \label{fig:plot_min_distance}
\end{figure}

Fig.~\ref{fig:plot_max_cost} shows how the maximum cost experienced along a
path given by the cost distribution changes as a function of the set speed of
the robot. The smaller the maximum cost, the safer the path. The figures shows
that as the speed increases, the maximum cost experienced by a robot being
planned using Dodger decreased. This is because when the robot is able to move
faster, it can maneuver through areas that only have a low cost for a small
period of time before becoming high cost areas again. Also, as the speed
increased over 2 $m/s$, there was a greater difference between the maximum
costs for paths through experimental configurations with small amounts of noise
and the maximum costs for paths through high noise configurations.  This is due
to the fact that no matter how fast the robot is able to move, the more noise
in a scene, the more the initial path will not represent the actual costs
through the environment and the planner will need to replan. The robot may move
to an area in which it thinks it will be safe, but if the obstacles deviate
from their prescribed trajectories, this area may not be safe any longer and a
new path is needed and therefore the cost for the path a robot is executing
will increase.  The paths generated by the potential fields planner did not
have a similar behaviour as the amount of noise increased, however, the maximum
cost was greater for all speeds than that of the paths generated by Dodger.
This is because the potential fields planner does not take into account the
trajectories of the obstacles when planning and is purely reactive. This plot
shows that having access to information about where obstacles are moving can
decrease the costs of the generated paths which in turn leads to safer
trajectories for the robot.

\begin{figure}[h!]
    \centering
    \includegraphics[width=0.48\linewidth]{figs/planner_mean_max_cost_0}
    \includegraphics[width=0.48\linewidth]{figs/pf_mean_max_cost_0}
    \caption{}
    \label{fig:plot_max_cost}
\end{figure}

Fig.~\ref{fig:plot_avg_cost}

\begin{figure}[h!]
    \centering
    \includegraphics[width=0.48\linewidth]{figs/planner_mean_avg_cost_0}
    \includegraphics[width=0.48\linewidth]{figs/pf_mean_avg_cost_0}
    \caption{}
    \label{fig:plot_avg_cost}
\end{figure}

\subsection{Variance}

\begin{figure}[h!]
    \centering
    \includegraphics[width=0.48\linewidth]{figs/planner_std_min_distance_0}
    \includegraphics[width=0.48\linewidth]{figs/pf_std_min_distance_0}
    \caption{}
    \label{fig:plot_std_min_distance}
\end{figure}

\begin{figure}[h!]
    \centering
    \includegraphics[width=0.48\linewidth]{figs/planner_std_max_cost_0}
    \includegraphics[width=0.48\linewidth]{figs/pf_std_max_cost_0}
    \caption{}
    \label{fig:plot_std_max_cost}
\end{figure}

\begin{figure}[h!]
    \centering
    \includegraphics[width=0.48\linewidth]{figs/planner_std_avg_cost_0}
    \includegraphics[width=0.48\linewidth]{figs/pf_std_avg_cost_0}
    \caption{}
    \label{fig:plot_std_avg_cost}
\end{figure}

\section{Computational Time}

\begin{figure}[h!]
    \centering
    \includegraphics[width=0.48\linewidth]{figs/planner_mean_times_0}
    \includegraphics[width=0.48\linewidth]{figs/pf_mean_times_0}
    \caption{Plots showing how the computational time changes as the speed
    increases for various amounts of obstacle position uncertainties}
    \label{fig:plot_comp_time}
\end{figure}

\begin{figure}[h!]
    \centering
    \includegraphics[width=0.7\linewidth]{figs/planner_small_mean_times_0}
    \caption{Description}
    \label{fig:plot_small_comp_time}
\end{figure}

\subsection{Variance}

\begin{figure}[h!]
    \centering
    \includegraphics[width=0.48\linewidth]{figs/planner_std_avg_times_0}
    \includegraphics[width=0.48\linewidth]{figs/pf_std_avg_times_0}
    \caption{}
    \label{fig:plot_std_comp_time}
\end{figure}

\section{Behaviour}

\end{document}
