
\begin{document}

\chapter{Introduction}

\label{chapter:introduction}

Path planning is a very important problem in robotics and computer science. It
is the problem of generating a path through an environment that if followed by
a robot, would move the robot from some initial configuration to a goal
configuration without coming into collision with any obstacles on the way. This
problem may seem easy for a human to solve, just walk around the obstacles and
get to the goal, but for a robot it can be extremely difficult. As humans, we
have amazing perception abilities and an unparalleled ability to assess the
risk we perceive and plan around it. These capabilities are not as developed in
artificial intelligence. Humans can walk around environments which are dynamic
and uncertain and (usually) reach wherever they are going without running into
moving obstacles or hitting walls. This is because as humans, we can determine
automatically where things are going to be in the future by looking at their
past locations and use this information to build safe, collision free paths to
our goals. Imagine you are driving a car and reach an intersection, you stop
because you see a car coming from the right. You have a choice to either move
forward crossing over the future path of the other car, or to wait for the car
to pass. By judging how far the car is away from you and how fast it is moving,
you can automatically determine whether or not it is safe to cross the road.
Likewise, imagine you are a waiter in a busy, hectic restaurant. You have to
bring an order to hungry customers. You are able to bring the customers the
food by predicting where other waiters and customers are going to be whilst you
move through environment in order to avoid spilling the food and sacrificing
your tip. Our brains do this planning and prediction automatically in order to
generate safe paths through stochastic dynamic environments. The aim of this
project is to use given information about the future motion of obstacles by an
external system in order to generate safer paths than state of the art planners
that have been designed for and operate in dynamic environments.

There has been outstanding progress by the robotics community to develop
algorithms that are able to plan the motion of a robot in order for it to reach
its goal. From this community, three different paradigms for path generation
and motion planning have emerged, geometric algorithms, reactive algorithms,
and sampling based approaches~\cite{choset, lavalle}. Geometric algorithms such
as the bug algorithm~\cite{weir} or visibility graph algorithm~\cite{vis}, use
the geometry of the environment to create exact geometric paths to the goal.
These algorithms almost always have no random component, and for a given
environment, will return the same path every time. The second paradigm for
motion planning, reactive algorithms, move the robot to the next best location
at the current time for a given sensing radius. These approaches are vastly
dominated by the use of potential fields in order to determine the direction
that a robot should move. The approaches use a combination of an attractive
potential function to guide the robot towards a goal and a repulsive potential
function that keeps the robot from coming into contact with obstacles. The
robot moves forward in time by determining the direction it should move that
would minimize the combined potentials. Sampling based approaches approximately
discretize the environment, also known as the search space, in order to
describe its connectivity by sampling plausible configurations and how to move
between them.  With the discretization, usually in the form of a graph or a
tree, paths to the goal are extracted using shortest path algorithms which seek
to minimize a given objective function. These approaches are becoming
exceedingly popular due to their running time and how they are able to scale
for high dimensional systems.

This project aims to utilize developments in motion planning, particularly
sampling based motion planning in order to move a robot safely from an initial
configuration to a goal configuration in a stochastic dynamic environment by
leveraging information about the trajectories of dynamic obstacles. By having
some idea about where the obstacles are going to move in the future, it is
possible to use sampling based techniques that can sample collision free and
low risk paths to the goal in space-time. The goal of this project is to create
algorithms that can provide low cost paths to the goal by utilizing the
information available about how the obstacles in an environment will move. This
work introduces a novel representation of dynamic obstacles and a novel
algorithm for searching the environment in space-time.  Proofs are also
provided that can guarantee the completeness of the search such that the
algorithms will always provide a path to the goal. This problem is important
because if robots are interacting with humans, the robot must move safely in
order not to come into contact with and hurt humans as well as minimize the
damage that can possibly occur to itself. Likewise, imagine a situation where a
robot is deployed to an environment for a long period of time. By generating
safe paths (i.e. paths that have an insignificant chance of colliding with an
obstacle), the robot can be deployed for longer periods of time without
maintenance, because it would be less likely that it would get damaged as a
result of a collision. A solution to this problem can provide safer paths for
the operating environment and the robot by leveraging information that can be
extracted about where obstacles are moving.  Likewise, this problem is
important because there exists systems that are able to predict the motions of
obstacles, however there is a lack of systems that use this knowledge to
generate safe paths.

\end{document}
