
\begin{document}

\chapter{Context Survey}

\label{chapter:contextsurvey}

For years the planning community has been developing algorithms that allow
robots to navigate through environments to reach a goal. As research
progressed, so did the complexity of the environments that the robots needed to
move through. It started with static obstacles which remain stationary for the
entirety of the plan, then on to dynamic obstacles with stochastic motion, to
dynamic obstacles that have predictable paths that planners can exploit. For
the last situation, a number of planners implementing different techniques have
been created. Techniques include evolutionary computation on geometric
structures in space-time to generate paths, safe interval path planning,
sampling based space-time search using collision prediction, and probabilistic
methods.

\section{Evolutionary Algorithms}

Wang et al.\ used a genetic algorithm that would generate a path around the
static and dynamic obstacles~\cite{wang2007mobile}. The obstacles were encoded
by a set of vertices in a polygon. The chromosomes used in the genetic
algorithms represented the paths that the robot can take to move to the goal
where the path is comprised of the initial and goal points along with obstacle
vertices.  The path generated for the robot would cling to the perimeter of an
obstacle, until it could move towards the goal similar to the bug
algorithm~\cite{weir}. The time at which the robot would reach each node was
also encoded into the path. Moving obstacles were represented as static
obstacles for a given time. Upon a sensed change in the environment such as a
dynamic obstacle veering off its prescribed trajectory, replanning using the
genetic algorithm was used to regenerate a collision free path.

Smierzchalski and Michalewicz have also used an evolutionary approach where the
chromosome represents the spatio temporal path in order to plan the movements
of a ship in a harbour where the obstacles are other ships in a shipping
lane~\cite{smierzchalski2005path}. The higher the evaluation for the chromosome
(i.e.\ the path) the safer and shorter the path is. Unlike the work done by
Wang et al.\, the path can be comprised of any point in the search space.  The
obstacles were represented geometrically and the dynamic obstacles were
represented by their initial configuration and velocity.  Since the path
represented in a spatio-temporal fashion, the genetic algorithm is able to
determine if a path will collide with a dynamic obstacle by deducing where the
dynamic obstacle will be at each time step along the path and checking for a
collision.  Replanning using the evolutionary algorithm was implemented to
account for obstacles deviating from their prescribed trajectories.

\section{Safe Interval Path Planning}

\section{Sampling in Space-time}

\section{Probabilistic Methods}

\end{document}
