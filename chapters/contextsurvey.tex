
\begin{document}

\chapter{Context Survey}

\label{chapter:contextsurvey}

For years the planning community has been developing algorithms that allow
robots to navigate through environments to reach a goal. As research
progressed, so did the complexity of the environments that the robots needed to
move through. It started with static obstacles which remain stationary for the
entirety of the plan, then on to dynamic obstacles with stochastic motion, to
dynamic obstacles that have predictable paths that planners can exploit. For
the last situation, a number of planners implementing different techniques have
been created. Techniques include evolutionary computation on geometric
structures in space-time to generate paths, safe interval path planning,
sampling based space-time search using collision prediction, and probabilistic
methods. Note that this context survey only covers approaches that make use of
models of obstacle behaviour and discards approaches that purely react to
obstacle motion.

\section{Evolutionary Algorithms}

Wang et al.\ used a genetic algorithm~\cite{galletly1992overview} that would
generate a path around the static and dynamic obstacles~\cite{wang2007mobile}.
The obstacles were encoded by a set of vertices in a polygon. The chromosomes
used in the genetic algorithms represented the paths that the robot can take to
move to the goal where the path is comprised of the initial and goal points
along with obstacle vertices.  The path generated for the robot would cling to
the perimeter of an obstacle, until it could move towards the goal similar to
the bug algorithm~\cite{weir}. The time at which the robot would reach each
node was also encoded into the path. Moving obstacles were represented as
static obstacles for a given time. Upon a sensed change in the environment such
as a dynamic obstacle veering off its prescribed trajectory, replanning using
the genetic algorithm was used to regenerate a collision free path.

Smierzchalski and Michalewicz have also used an evolutionary
approach~\cite{evosurvey} where the chromosome represents the spatio temporal
path in order to plan the movements of a ship in a harbour where the obstacles
are other ships in a shipping lane~\cite{smierzchalski2005path}.  The higher
the evaluation for the chromosome (i.e.\ the path) the safer and shorter the
path is. Unlike the work done by Wang et al.\, the path can be comprised of any
point in the search space.  The obstacles were represented geometrically and
the dynamic obstacles were represented by their initial configuration and
velocity.  Since the path represented in a spatio-temporal fashion, the genetic
algorithm is able to determine if a path will collide with a dynamic obstacle
by deducing where the dynamic obstacle will be at each time step along the path
and checking for a collision.  Replanning using the evolutionary algorithm was
implemented to account for obstacles deviating from their prescribed
trajectories.

Dunwei and Na used particle swarm optimization (PSO)~\cite{kennedy2010particle}
to generate local paths around dynamic obstacles that are represented by
trajectory bands~\cite{dunwei2011local}.  In order to account for the
stochasticity in the obstacle trajectories, a distance around each obstacle's
trajectory was used to describe how much the obstacle can deviate from it.
Using these bands, an objective function was created that represents the safety
of the generated path. PSO was used to generate a path that could minimize the
objective function.

\section{Safe Interval Path Planning}

Narayanan et al.\ and Phillips et al.\ have developed algorithms for generating
safe paths to guide a robot to a goal configuration by determining time
intervals for which it would be safe to travel through certain areas in the
search space and using a graph search algorithm to find a path to the goal that
only intersects safe time intervals~\cite{asipp, sipp}.  Safe time intervals
are determined by iterating through the trajectory of each obstacle and
updating a query structure that can be used to check whether a certain space
will be free for a given period of time. A lattice structure is used to
discretize the search space and an A* variant, Anytime Reparing A*
(ARA*)~\cite{likhachev2003ara} is used to search this lattice. This algorithm
relies on the robot waiting at certain locations for areas to become free of
obstacles before it is able move towards the goal.

\section{Sampling in Space-time}

Berg et al.\ have developed a sampling based approach that generates a
probabilistic roadmap of the same dimension as the configuration space of the
robot with time that takes into account static and dynamic obstacles.
\cite{van2006anytime}. A novel graph search technique similar to D*
Lite~\cite{koenig2002d} was implemented that can automatically repair the path
when new information about obstacle motion is available such that the robot
will not be led into a collision with an obstacle. Obstacles are represented by
elongated objects in space-time and their stochasticity is represented by their
size as the time away from the current time increases.

Hsu et al.\ have also created an algorithm that generates a probabilistic
roadmap in space-time where each node represents the configuration of the robot
and a time at which the robot would be at the
configuration~\cite{hsu2002randomized}. Using the probabilistic roadmap, a
shortest path algorithm is used to find the fastest route from the initial
configuration of the robot to the goal configuration. Obstacles are represented
in a similar fashion as the work done by Berg et al.~\cite{van2006anytime}
except there is no representation of stochasticity. The planner handles updated
information about the dynamic obstacles by updating the probabilistic roadmap
and researching the graph for a path to the goal.

\section{Probabilistic Methods}

Jensen et al.\ developed an algorithm that uses a probabilistic obstacle model
to represent and employs a gradient decent method to find path to the
goal~\cite{jensen2003motion}.  A gradient decent approach tries to minimize the
probability of the robot coming into collision with an obstacle along the path.
This method is general but assumes that the robot has a sensing radius and is
able to move in the direction that has the lowest probability of collision.
Likewise, the paper shows that the combined probability model for the obstacles
in the scene only has one global minima. The algorithm can also be adapted so
that there is a trade off between the probability of collision and the length
of the path.

Rodriguez et al.\ uses a dynamic probabilistic roadmap to find a global path to
the goal and local kinodynamic motion planning to evade dynamic
obstacles~\cite{rodriguez2007framework}. A self-repairing probabilistic roadmap
is generated that takes into account the location of static obstacles. Using
this roadmap, a global path to the goal is found using a graph search
technique. The global path represents a sequence of sub-goal regions and a
kinodynamic local planner using an approach similar to RRT~\cite{rrt}
determines the local path that moves the robot to each sub-goal region in
sequence. This approach allows obstacles which have been assumed to be
stationary to move or have uncertainty in their positions. The probabilistic
roadmap will automatically repair itself upon obtaining updated information
about the static obstacles such that no edge is in collision with an obstacle.

\section{Limitations of Related Work}

The main limitation of the developments described is that there is no solution
that uses the obstacles' known stochastic trajectories to build probabilistic
representations and use these representations to build complete low cost paths.
Some of algorithms currently either assume that the planner only knows that the
obstacles will move randomly around a point~\cite{rodriguez2007framework} or
that the obstacles will move along a fixed trajectory~\cite{hsu2002randomized},
and if they deviate, a new plan is developed. Hsu et al.\ assumed perfect
information about the obstacle trajectories to build an initial path through
the environment, and if the information was incorrect or the obstacles deviate
from their prescribed trajectories, the planner develops a new route to the
goal. Berg et al.\ increased the size of the dynamic obstacles in space time to
account for random motion, but to plan around them, represented the obstacles
as static objects in space-time thus not devising a cost metric for their
motion. Likewise, the approaches that used safe interval path
planning~\cite{asipp, sipp}, did not build probabilistic models of the
obstacles, but instead determined intervals at which it would be safe to travel
through certain areas and is thus heavily reliant on the robot waiting at
certain locations. The evolutionary approaches described were geometric and
also did account for the stochasticity in the known trajectories of the dynamic
obstacles. The one approach that did create probabilistic models of the dynamic
obstacles~\cite{jensen2003motion} employed a gradient decent method similar to
that of potential fields that is described in Sec.~\ref{sec:costpf} to be
ineffective in certain situations and that a sampling based approach is more
complete. All of these approaches lack an elegant, sampling based algorithm
that can take into account the stochastic trajectories of dynamic obstacles in
order to rely less heavily on replanning, and to make use of the available
information from the environment.

\end{document}
